%% bare_conf.tex
%% V1.4b
%% 2015/08/26
%% by Michael Shell
%% See:
%% http://www.michaelshell.org/
%% for current contact information.
%%
%% This is a skeleton file demonstrating the use of IEEEtran.cls
%% (requires IEEEtran.cls version 1.8b or later) with an IEEE
%% conference paper.
%%
%% Support sites:
%% http://www.michaelshell.org/tex/ieeetran/
%% http://www.ctan.org/pkg/ieeetran
%% and
%% http://www.ieee.org/

%%*************************************************************************
%% Legal Notice:
%% This code is offered as-is without any warranty either expressed or
%% implied; without even the implied warranty of MERCHANTABILITY or
%% FITNESS FOR A PARTICULAR PURPOSE! 
%% User assumes all risk.
%% In no event shall the IEEE or any contributor to this code be liable for
%% any damages or losses, including, but not limited to, incidental,
%% consequential, or any other damages, resulting from the use or misuse
%% of any information contained here.
%%
%% All comments are the opinions of their respective authors and are not
%% necessarily endorsed by the IEEE.
%%
%% This work is distributed under the LaTeX Project Public License (LPPL)
%% ( http://www.latex-project.org/ ) version 1.3, and may be freely used,
%% distributed and modified. A copy of the LPPL, version 1.3, is included
%% in the base LaTeX documentation of all distributions of LaTeX released
%% 2003/12/01 or later.
%% Retain all contribution notices and credits.
%% ** Modified files should be clearly indicated as such, including  **
%% ** renaming them and changing author support contact information. **
%%*************************************************************************


\documentclass[conference]{IEEEtran}
\usepackage{cite}
\ifCLASSINFOpdf
  % \usepackage[pdftex]{graphicx}
  % declare the path(s) where your graphic files are
  % \graphicspath{{../pdf/}{../jpeg/}}
  % and their extensions so you won't have to specify these with
  % every instance of \includegraphics
  % \DeclareGraphicsExtensions{.pdf,.jpeg,.png}
\else
  % or other class option (dvipsone, dvipdf, if not using dvips). graphicx
  % will default to the driver specified in the system graphics.cfg if no
  % driver is specified.
  % \usepackage[dvips]{graphicx}
  % declare the path(s) where your graphic files are
  % \graphicspath{{../eps/}}
  % and their extensions so you won't have to specify these with
  % every instance of \includegraphics
  % \DeclareGraphicsExtensions{.eps}
\fi

\hyphenation{mix-networks net-works}

\begin{document}
\title{Catshadow Mix Network Threat Model}
\author{\IEEEauthorblockN{David Stainton}
\IEEEauthorblockA{}
}
\maketitle

\begin{abstract}
We show that the catshadow decryption mix network messaging system
provides a stronger threat model than even Signal used with Tor.
In particular we show that decryption mix networks must deliver messages
in at least two route segments where one of them is not controlled by
the sender, in order to provide the security properties suitable for
high risk users such as whistleblowers and journalists.
\end{abstract}
\IEEEpeerreviewmaketitle

\section{Introduction}
I herein shall describe a mixnet messaging system which endeavors to
not only provide users with usual cryptographic assurances such as
message confidentiality, integrity and, authenticity... but we have
the additional goals of reducing the amount of metadata leaked to
passive network observers and network operators. In cryptographic
messaging systems like Signal and Wire, users leak their entire social
graph to the operators of the network. Additionally, other important
kinds of metadata are leaked such as message send time, message
receive time, message size. All of these are available to the system
operators without breaking any of the cryptographic systems used.

Catshadow was inspired by the Loopix paper \cite{piotrowska2017loopix}
but differs in the privacy notions, threat model and message delivery
scheme. The Loopix messaging system does not hide contact network
locations from one another but instead message delivery is done using
only forward Sphinx packets and therefore the sender knows the final
destination.

In the usual designs described in much of the mix network literature,
decryption mix networks function as end to end mix networks where the
sending client knows the final destination of the message and knows
what its ciphertext will look like. In contrast to this design, I 
shall describe a decryption mix network messaging system which
prevents contacts from learning one another's network location and
prevents clients from being able to predict the final ciphertext
retreived by the receiving client.

\section{Privacy Goals}
Informally our main goal is to prevent network operators and passive network observers
from learning who is communicating with whom. We achieve this privacy by means of the
follow privacy notions as described in \cite{notions-pets2019}:
\begin{itemize}
\item Sender Unobservability
\item Receiver Unobservability
\item Sender Receiver Unlinkability
\end{itemize}
This design retains the notion of Sender Receiver Unlinkability even
with an adversarial sender.

This analysis is only applicable for online users since offline users are
clearly not sending any messages. Furthermore, Sender Unobservability is achieved
by means of clients sending decoy traffic so that network observers will not know
when a legitimate message is sent.

\section{Adversarial Model}
Mixnets provide the Anytrust property which means that if a route
consists of at least one honest mix then there is still uncertainty
for a passive network observer for the correlation between sent and
received messages. This of course is predicated on the assumption that
the mixnet has sufficient traffic from a sufficient number of users.
Conversely, if an adversary has compromised the entire route it is then
defined to be a bad route and implies immediate correlation between sent
and received messages.

Given that Katzenpost is designed for the asynchronous messaging use
case, it's remarkable that queueing messages at the edge of the
network also implies that classical intersection attacks with full
granularity involve compromising one or more Providers whereas most of
the mixnet literature on the topic assumes different situation where
message flows to clients are visible by passive network observers. In
Loopix and Katzenpost this is not the case because of the queueing and
the traffic padded protocol used by clients to retreive messages.

The adversary needs to compromise the Providers in order to learn
which client's message queue received a given message. Without this
information, intersection attacks would take much longer because
Providers can have many client message queues which would make passive
network observations contain a low amount of statistical
information. We can therefore say that even if the adversary
compromises the sending Provider and the receiving Provider that this
would only allow the adversary perform an intersection attack and
would not immediately allow linking senders with the receivers if the
mix network had enough users and enough traffic.

\section{Sphinx}
The above goals are achieved by adding delay with a mix strategy and
by some of the properties of the Sphinx cryptographic packet format
\cite{DanezisG09} \cite{SphinxSpec} where forward Sphinx packets are
used along with Single Use Reply Blocks (SURB), an anonymous delivery
token.  The delivery of messages is due to the composition of two
routes. The first route is selected by the sending client using a
forward Sphinx packet. The second route is selected by the receiving
client with a SURB reply Sphinx packet. In between these two routes
the message dwells in a queue on one of the Providers. In other words:
Alice sends a message to Bob's remote queue which is hosted by one of
the Providers. Sometime later Bob retreives the message from his
remote queue by sending a forward Sphinx packet destined to his
message queue service on the remote Provider. The payload of his
Sphinx packet contains a valid query message AND a SURB which allows
the Service to send an anonymous reply to the client.

The mix network is composed in a stratified topology
\cite{topology-pet2010} where the topology is published in the PKI
network consensus document. These topological routing restrictions are
enforced by the way in which mix servers use the Katzenpost cryptographic wire
protocol. \cite{KatzMixWire} That is to say, if a mix tries to violate
the topological restrictions the cryptographic wire protocol authentication 
will fail thus preventing the connection from being made. This cryptographic 
wire protocol is also used for the interactions of the other Katzenpost mix
network entities such as client and authority servers.

Clients compose Sphinx packets where the payload is end to
end encrypted (albeit in a nested manner) to the destination
Provider. However the destination Provider sends it's payload back to
the client using the SURB, and this is end to end encrypted from the
sending Provider to the client. Therefore the receiving Provider of
this SURB reply only observes ciphertext because only the receiving
client has the vector of keys which can decrypt the SURB reply
payload.

\section{Mix Network PKI}
The mixnet PKI system \cite{KatzMixPKI} is the root of all
anonymity and security dependencies in the system. Unlike Tor,
Katzenpost does not allow mixes to join the network
automatically. Instead each mix identity public key is whitelisted in
the PKI configuration file to allow mixes to participate in the
network. I will describe various attacks against the PKI in a
proceeding section of this document. The Katzenpost PKI works very
similarly to the Tor and Mixminion Directory Authority system.

Katzenpost \cite{KatzMixnet},\cite{KatzEndToEnd},
\cite{SphinxSpec} borrows some designs from The Loopix Anonymity
System \cite{piotrowska2017loopix}, such as the Sphinx cryptogaphic
packet format \cite{DanezisG09}, the Poisson mix strategy
\cite{stop-and-go} and, the stratified topology
\cite{topology-pet2010}. However Katzenpost differs in a number of
ways from Loopix.

\section{The Poisson Mix Strategy}
Katzenpost uses a continuous time mix strategy called the
Poisson mix strategy. As described in the Loopix paper, all clients
use the same set tuning parameters to configure their Poisson
processes which the client scheduler uses to determine delays between
sending decoy and legit traffic: $\lambda_p, \lambda_l, \lambda_d,
\mu$ Katzenpost uses it's PKI to distribute these tuning parameters to
the mixes and clients. However, catshadow in it's current design does
not use the $\lambda_d$ parameter. If clients sent traffic other than
loops there would be a need for drop decoy messages. The current
design uses only loop decoy messages because clients only send queries
which should result in a response sent via a SURB supplied by the
client's query message. Possible modifications to this client
scheduler design are discussed in the section on statistical
disclosure attacks.

Clients also use a Poisson process to select the delay of each hop of
the routes used by their forward Sphinx packets and SURB replies. The
client therefore knows all the mix delays in the full round trip of
the query and response. It is useful for the client to possess this
round trip mix time information and we shall discuss it in a later
section when we discuss achieving reliability via an Automatic Repeat
reQuest protocol scheme. If tuned correctly a mix network using the
Poisson mix strategy can maintain a given level of mix entropy by
increasing or decreasing the mix delays in relation to frequency of
sending decoy traffic. That is to say, increasing mix delays and
increasing the frequency of decoy traffic both result in increased mix
entropy. To drive the point home about anonymity gaurantees or lack
thereof we can say that the Poisson mix strategy does not guarantee a
particular mix entropy because when fewer users are using the mix
network the entropy on all the mixes decreases. Furthermore the tuning
and the resulting mix entropy is dependent on a myriad of factors
besides the Loopix tuning parameters we mentioned for tuning the
poisson processes. These additional factors include the number of
users currently participating in the mix network, number of mixes,
number of topology layers, number of Providers, distribution of users
among the Providers etc.

Mix networks in general do not provide much protection against various
types of traffic analysis if they do not have enough users using
them. The Poisson mix strategy in particular makes an additional
tradeoff against protecting users and does not guarentee a specific
mix entropy but instead uses tuning parameters which are tuned to a
specific range of participating users and for various other contraints
that may not always be met. It is remarkable that the Mixminion mix
network used the pool mix strategy which always guarantees egress
messages in the mix are always selected from a pool that maintains a
set threshold of messages where N number of messages are removed from
a pool when the threshold is exceeded by N. This guarantees all your
messages will be mixed with enough entropy assuming a high enough
threshold is selected and there are enough users. However the pool mix
makes this guarantee with a huge performance trade-off where message
latency can be very high. This presents a useability problem that
would likely be a turn off for many users. Can the Poisson mix
strategy be improved such that it receives dynamic tuning based on the
number of users?  Or is there a different mix strategy which would be
safer to use than the Poisson mix but higher performance than the pool
mix strategy?

Katzenpost aims to be a general purpose secure messaging transport
which can be used to compose various messaging systems for multiple
applications. Katzenpost also has a particular usage of Sphinx SURBs
that is not found anywhere else. Instead of allowing users to exchange
SURBs with a Nymserver \cite{minion-design} to send each other messages
where anonymous replies are possible, we instead elect to have clients
send SURBs to mixnet services so that a query response and be routed
to the client without exposing the client's network location.

Clients interact with services on the mix network which run on mixes
we designate as Providers. These Providers occupy the first and last
of the layers in the stratified network topology where all the mixes
in a given layer are only allowed to send messages to the mixes in the
next topology layer. In Katzenpost this topology is strictly enforced
by the Noise transport protocol authentication \cite{KatzMixnet}
\cite{KatzMixWire}. That is, each mix only allows inbound connections
to send it packets and inbound connections are only allowed from the
previous layer in the topology.

\section{Catshadow}
The catshadow usage of the Katzenpost framework is rather simple
and only involved customizing two components: a server plugin to
run the remote message queue on the Providers and a custom mixnet
client.

In catshadow, the retrieval of messages from the remote message queue
uses a special feature of the Sphinx cryptographic packet format
called Single Use Reply Blocks. SURBs are essentially a short lived
delivery token allowing anonymous replies. In Katzenpost, SURBs are
generated by clients and sent along with a query to a service running
on a Provider. The service can then use the SURB to send it's query
response. Catshadow clients use this simple strict query response SURB
based protocol to talk to a remote message queue. This gives clients
strong location hiding properties from one another.

All communication to and from participants in the Katzenpost Mix
Network is done via the Katzenpost Mix Network Wire Protocol
\cite{KatzMixWire}. The client generates a link key pair and use it
to connect to a randomly choosen Provider in the set of all Providers
in the network. This information must first be gathered by querying
the mixnet PKI for a consensus document.  Providers allow clients to
connect with any key however the keys will be garbage collected if
unused for more than one hour. This allows for client disconnects and
reconnects such that clients will be able to retreive messages queued
locally on the Provider. This requires the client to randomize it's
selection of Providers only after some threshold down time duration or
if switching physical locations and thus access to the Internet.

Upon first initialization, clients connect to the mix network and then
select a random Provider and create a remote spool on it using Sphinx
SURBs to interact with the remote spool service plugin. Although the
client's queries to the remote service are done with a forward Sphinx
packet whose end to end encryption terminates on the destination
Provider, the replies using the client generated SURB are encrypted
using a key that only the client knows. Two clients, Alice and Bob
after intializing their clients with remote queues they can initiate
a PANDA exchange. They use PANDA to exchange not only key material but
their queue location information. They each check their own queue for
messages from other contacts. Once the keys are exchanged, a triple
diffie hellman variant is used to initialize both client's double
ratchets. Each client uses one double ratchet per contact and
therefore performs trial decryption on all messages received from
their remote message queue.

\section{Denial of Spam}
In general it is not possible to receive spam in this messaging system.
However two clients may confirm they are speaking to the same contact
if they share their contact queue location information. Queue location
is designated to be a queue identity number and a Provider name string.
A user can receive messages from unwanted clients however unless their
has been a prior key exchange the messages will fail to decrypt.

\section{Identity Leakage to Clients}
Clients that share contact information can determine the intersection
of their contacts by comparing queue identity information.

\section{Metadata Leakage to Providers}
The Provider of a client's remote message queue doesn't immediately
learn which client appended each message. However it is possible for
an adversary in this position to perform long-term statistical
disclosure attacks which correlate received messages to the sets of
users online at the time the messages were received. This assumes the
adversary has compromised the Provider and can watch the network
traffic on the other Providers. See section on statistical disclosure
attacks.

\section{Denial of Service}
Catshadow does not prevent someone from flooding the mixnet with
Sphinx packets. Although there is optional per client rate limiting on
the Providers, an adversary wishing to denial-of-service the network
can easily generate many client keys and initiate many connections to
the Providers to increase their rate of packets. Furthermore, the SURB
based query response protocol amplifies such attacks by a factor of
two.

There are additional denial of service attacks, such as creating too
many accounts on each Provider and, filling up the local and remote
message queues. The Directory Authority PKI for Katzenpost
\cite{KatzMixPKI} can also be made to deny service. In particular the
current version does not use a byzantine fault tolerate protocol and
can therefore be prevented from generating a consensus file in each
voting round.  This would effectively DOS the whole mix network
because without a consensus file to distribute key material, the
network cannot be used.

\section{Classical Mix Network Attacks}
There are five major categories of classical decryption mix network attacks
and they are:

\begin{enumerate}
  \item n-1 attacks \cite{trickle02}
  \item epistemic attacks
  \item compulsion attacks \cite{ih05-danezisclulow}
  \item tagging attacks
  \item statistical disclosure attacks
\end{enumerate}


\subsection{n-1 Attacks}
The poisson mix strategy has a very simple n-1 attack where the
adversary drops or delays all input messages to a given mix but let's
the target message enter once the mix is probably empty. The adversary
simply waits until the mix routes the message to the next hop. This
amounts to a sort of denial of service attack that results in being
able to trace one packet through the network when the n-1 attack is
repeated on each mix in the route until the entire route is learned.

Katzenpost mixes can optionally send loop decoy Sphinx packets,
however no defensive reaction is trigged by the loss of these loop
messages as has been previously described \cite{piotrowska2017loopix},
\cite{danezis:wpes2003}. We are forced to use an imperfect heuristic
to decide when loss of mix loops implies intentional packet loss for
the purpose of orchestrating an n-1 attack or if it's benign or
performance related packet loss. We plan to implement this heuristic
detection and partial defense in the future. As it stands now
Katzenpost has no defense against n-1 attacks.

\section{Epistemic Attacks}
Epistemic attacks \cite{DanezisC06}, \cite{esorics05-Klonowski},
\cite{danezis-pet2008} can occur when the adversary learns some
information about a client's unique view of the network. Client path
selections can be fingerprinted and are easily recognizable if they
use a specific subset of available mixes. Therefore to prevent this
attack we designed our PKI \cite{KatzMixPKI} to distribute the exact
same consensus document to each client for a given epoch time
duration.

\section{Statistical Disclosure Attacks}
Statistical disclosure attacks work to some extent on all anonymous
communication networks. Catshadow is designed to provide partial
defense against long-term intersection attacks as well as sufficient
defence against short-term timing correlation attacks.

The classical mix network literature has described intersection
attacks in terms of a mix network where a passive network observer can
watch individual clients receive messages. This assumption can be
otherwise stated that the adversary observers all the inputs and
outputs of the mix network and thus receives a high granularity of
statistical information. However the Loopix design makes use of
message queues on the edge of the network so that messages can be
received asynchronously.  It turns out that this design also reduces
accuracy of the stastical information available to passive network
observers who will only see a specific Provider received a Sphinx
packet but not know which client queue received it. That is to say,
the adversary must compromise the receiving Provider in order to
determine which users receive a given message. However there will
certainly be longer term statistical information leaked to passive
adversaries, especially if user behavior is repetative and
predictable.

Catshadow forms bidirectional communication channels by combining
unidirectional communication channels from two clients. Each of these
unidirectional channel is a message queue operated by one of the
Providers on the mix network. Users select a random Provider to
connect to and herein lies a dilemma. If cleints connect to all
Providers but the one they use to host their message queue, then these
connections will eventually leak which Provider they are using to a
passive network observer who is able to observe all of a client's
interactions with the mix network. A client who changes their location
before connecting has some defense against this.

The other aspect of this dilemma is if we choose to allow clients to
select any Provider to connect to even if it happens to be the Provider
they use to host their message queue then that Provider can perform
stastical disclosure attacks to clients with message queue queries.
If this linkage is established then the Provider learns the clients
network location, unless Tor is used. Katzenpost allows Providers to
optionaly only advertise their network location as a Tor onion service.
Tor onion service usage is not adequate location hiding defense for our
adversarial model because of being broken in a mere few seconds by a sufficiently
global adversary, however I mention it here to indicate the attack would
then proceed to break Tor to find the client's location.

Clients retreive messages by sending SURBs bundled in Sphinx packets destined
to the Provider hosting their message queue. In this manner clients must
periodically poll for new messages. Catshadow currently uses the FIFO queue
scheduler as described in \cite{piotrowska2017loopix}. This however leaks
statistical information in that messages sent to all of the Providers
are not uniformly distributed. A passive adversary who collects enough
information will be able to determine which Provider corresponds to a
specific client for hosting it's message queue.

One possible solution I have not yet implemented for catshadow would
be to enforce uniform distribution of messages among all
Providers. That is, the client would send decoy loops to each Provider
in equal portions except that the Provider it's polling would receive
somewhat fewer decoy messages. This would necessarily add more
latency on the processing time of the client's egress FIFO queue. The
latency would grow linearly with the number of Providers on the
network. Another somewhat equivalent solution would be to use a
separate Poisson process to generate delays for processing one egress
FIFO queue per destination Provider.

\section{Tagging Attacks}
The Sphinx cryptographic packet format \cite{DanezisG09} allows
for a one bit tagging attack under certain circumstances. The reason
for this security compromise is to allow for the design of the Single
Use Reply Block. The Sphinx header is MAC'ed but the packet body is not.
Instead, the body is encrypted with a wide-block cipher (an SPRP). This
ensures that an expected tag in the beginning of the plaintext can be
used to verify the plaintext of the final decryption. Therefore in order
to make use of this to perform a tagging attack, the adversary must have access
to the result of the final SURB reply decryption as well as the ability to
tag the packet some number of hops earlier in the route.

Without compromising any client devices there does exist a tagging
attack in catshadow: If the adversary has compromised the Provider
that a given client is interacting with, a tagging attack can be used
to confirm that a given Sphinx packet sent by the client is the same
packet received by a specific Provider. By flipping one or more bits
in the Sphinx packet body, the adversary ensures that the Sphinx
decryption on the destination Provider will fail due to not finding
the expected authentication tag in the plaintext. However this attack
is made more difficult due to the link encryption between nodes
in the Katzenpost mix network. \cite{KatzMixWire} Without breaking the
link crypto the adversary would have to compromise the client's
Provider in order to flip a bit in the body of the client's Sphinx
packet. Catshadow chooses Providers at random upon startup unless
recently disconnected.

\section{Compulsion Attacks}
Compulsion attacks are a problem for
decryption mix networks since the cryptographic transformation of the
Sphinx packets are used to form our communication channels we do not
have the forward secrecy properties using new key exchanges in both
directions. Instead we achieve partial forward secrecy by mix key
rotation and redistribution of mix keys via the mixnet
PKI. \cite{KatzMixPKI} This effectively limits the use of a mix key if
it is compromised.

The other defense used in Katzenpost is to use link encryption between
components in the mix network. Our Noise based protocol
\cite{KatzMixWire} uses a post quantum hybrid forward secret handshake
pattern to make it difficult for adversaries to capture a Sphinx
packet for use in a compulsion attack.

The usage of SURBs make Katzenpost and catshadow vulnerable to
compulsion attacks if the adversary compromises the Provider which
receives a SURB from a client.  In that case, the adversary would
learn the first hop from the SURB and then compell that mix operator
into disclosing their mix key. The adversary uses the mix key to
decrypt the Sphinx header component of the SURB and then learns the
next hop. The compulsion attack continues until the destination
Provider and queue identity is discovered by the adversary. After
performing this Sphinx packet compulsion attack the adversary must
link the Provider and queue identity to a specific user in order to
learn the receiving client's network location (e.g. home IPv4
address).

Although there are other defenses \cite{ih05-danezisclulow} we'd like
to implement in the future, the primary defenses for this attack are
Katzenpost Noise based PQ hybrid link transport protocol
\cite{KatzMixWire} and mix key rotation \cite{KatzMixPKI}. Another
possible future defense are forward secret mixes \cite{Dan:SFMix03}
which use an alternate key once they decryption a client
identity. This allows the mix to destroy the mix key immediately after
use however it also leaks information to the mix about which entity is
sending the packets. This security versus metadata leakage tradeoff
should be carefully examined in terms of the overal system design to
determine if it's a design worthy of inclusion.  At this time there
are no plans to include forward secure mixing in Katzenpost.

In should be clear that selecting organizing the topology into several
MLATs and in geographically distant locations would in theory increase
the difficulty in performing a compulsion attack using legal means. I
expect that powerful groups such as the NSA would be willing to
illegally compromise all the mixes in the network or all the mixes in
a specific route they are interested in pursuing. To do this I assume
that the NSA stockpiles zero-days for the common hardware and software
run by some or all of the mixes and have completely automated using
remote code execution vulnerabilities and escalating privileges with
additional attack payloads etc. Therefore the operational security of
the mixes and directory authorities are outside the scope the threat
model described in this document. However the Katzenpost software
project should try to implement some features that make additional
security hardening easier.  Katzenpost does not at this time support
usage with a hardware security module.  Mix keys are deleted some time
after they expire and the current epoch duration is set the three
hours.

\section{Compulsion Attacks via Path Selection}
Katzenpost uses the stratified topology and therefore each message a
client sends uses a newly selected random path through the network. If
the adversary has compromised a mix in each network topology layer
then eventually a client will select a bad route. In the mix network
adversary model a bad route is defined as a route in which each hop is
compromised by the adversary. This allows the adversary to link input
and output mix network messages. However catshadow communication channels
between contacts always involve two full routes through the mix network.
Therefore compromising a single route would not be sufficient to link two
clients on the network.

The strategy used by \cite{cryptoeprint:2016:489} and
\cite{cryptoeprint:2017:1000}, where a set of users uses cascade
published by the PKI for some fixed time duration, may be a better
probability tradeoff. On the other hand it may be a worse tradeoff
for clients that happen to select a bad cascade. I have yet to see a
probability analysis that justifies one topology strategy of the other.

\section{Confirmation Attacks}
I shall describe an active confirmation attack against Sphinx based an
end to end reliability protocol using an Automatic Repeat reQuest
scheme. Currently the catshadow messaging system does not provide end
to end reliability. I plan to add reliability and the described
partial defense against the active confirmation attack in the future.

This protocol feature exposes catshadow clients to an active
confirmation attacks by adversaries which have compromised a client's
message queue Provider. In this attack the adversaries goal is find
the Provider which the client is locally connected to. The adversary
divides the set of all Providers into two group and prevents one of
the groups from receiving the SURB reply from the queue service. In
this attack the adversary sees the client sequence number in the
message queue query and thus can learn which group the client was in
from subsequent messages. This attack can be performed in $\log_{2}N$
steps where N is the number of Providers in the network.

The partial defense against this attack is for clients to randomize
the retransmission delays above some threshold where it is assumed the
adversary will not be willing to cause outages for that length of
time. However at this time we do not have even a heuristic defense
against this attack designed or implemented. However it seems the mix
decoy loops can be used by Providers to learn when they are
potentially experiencing an outage as part of an active confirmation
attack. However unlike the heartbeat protocol design defense against
n-1 attacks mentioned earlier, in this case sending additional decoy
messages obviously does prevent the attack. However if the Provider
responded by denying service for some period of time this could delay
or prevent the attack. Clients can also use their decoy loops to learn
when there is a potential routing outage due to a potential active
confirmation attack.

If I expand the catshadow messaging system to include a SURB based
publish subscribe protocol where clients upload multiple SURBs to a
Provider in order to receive future messages for the given
subscription then this too will expose clients to an additional
attack. Upon receiving multiple SURBs from the client, the adversarial
Provider could immediately send many SURB replies using all of the
SURBs while simultaneously observing the received message rate for
each Provider. If the entropy introduced by the messages from other
users is low enough and the number of received SURBs is high enough
there is some non-negligible probability that all the adversary
replies will arrive at the client's Provider within some probable
window of time which would allow the adversary to gain a statistical
confirmation of the client's Provider.

\section{PANDA Attacks}
An early prototype version of Catshadow used the PANDA protocol \cite{PANDA} 
to facilitate not only key exchange but the establishment of bidirectional 
communications channels between communication partners. Later PANDA was 
replaced with the Reunion protocol.

\section{Reunion: an improved asynchronous anonymous PAKE protocol}
Reunion is better than PANDA for several reasons including not being
vulnerable to precomputation attacks by the server, leaking far less
metadata so we can actually hide the fact that a successful key exchange
has even occured. Attacks on the Reunion protocol are outside the scope
of this document.

\section{Public Key Infrastructure Attacks}
Katzenpost has a PKI implementation which uses a deterministic voting
protocol between multiple security domains and is inspired by the Tor
and Mixminion Directory Authority systems. The Katzenpost Directory
Authority uses a non-byzantine crash fault tolerant deterministic
voting protocol to generate the consensus document containing
signatures from a threshold number of authority protocol
participants. Each directory authority shares some configuration in
that they each must possess identical whitelists of the public
identity keys for each mix and authority in the mix network.

This current design can have each and every voting round sabotaged by
one or more bad acting protocol participants. It is my understanding
that using a modern byzantine fault tolerant protocol would solve this
problem.

The more important security problem here is that if
the adversary compromises K of N total protocol participants then they
get to decide the contents of the mix network consensus
document. Controlling the contents of this document can be used to
only advertise mixes controlled by the adversary for
example. Therefore a real world deployment of this system should make
sure K is sufficiently high to at least make the cost higher. Again we
should reiterate, the NSA and some other powerful organizations
probably stockpile zero-days, they may indeed have fully automated
exploiting various remote code execution vulnerabilities that can be
used to compromise mixes and authorities in a public deployment of
this mix network. It seems our only partial defense against these
attacks are to make them more noticeable and thus increase the risk to
the groups that would perform such attacks against an anonymous
communications network.

There are additional anonymity considerations with the Katzenpost PKI.
Katzenpost is a continuous time mix strategy meaning that messages do
not progress along their route through the network is fixed protocol
rounds. Therefore to faciliate smooth transition from one epoch to the
next where mix keys are rotated, there must be some grace period where
mixes perform trial decryption on received Sphinx packets to determine
which key to use, the old mix key or the new epoch's mix key.

Likewise there is a grace period for topology rerandmization which
leads to a temporary reduction in mix entropy. Mix placement in the
topology is randomized using a shared secret as a seed. Each authority
voting round also produces a new shared random value using a hash
based commit and reveal protocol. This topology rerandomization only
occurs when a threshold number of mixes per layer are removed from the
consensus document because their descriptors were not uploaded. That
is, mix outages are detected in the next voting round by the absence
of their descriptors.

Rerandomising the topology is helpful when the layered topology becomes
unbalanced because it can prevent a single mix operator from using
their mix as the only mix in a given layer. On the other hand we
shouod not rerandomize topology too often because it effectively
splits the anonymity set into two on each mix during the grace
period. Said another way, from the perspective of a passive adversary,
each input message originates from only one topology layer under
normal circumstances and therefore there is no distinguishing
characteristics among the messages which means there is logically only
one set of messages being mixed. However in the case where messages
originate from two different topology layers, this essentially splits
the anonymity set into two sets because the message source becomes a
distinguishable characteristic of each message for the grace period
duration in order to facilitate a smooth transition between epochs
which is what is needed by a continuous time mix strategy. This
prevents clients from needing to apply special cased scheduling around
epoch boundaries.

An adversary could prevent mixes to upload their descriptors to the PKI
and thereby prevent them from being included in the consensus document.
Additionally if the adversaries managed to cause some key outages it may
cause the topology to be rerandomized in the next voting round and that
reduced entropy could facilitate another type of attack.

\section{Conclusion}
We have systematically examined the attack vulnerabilities and
adversary model for the catshadow decryption mixnet messaging system
where the classical mixnet attack categories are:

1. n-1 attacks
2. compulsion attacks
3. tagging attacks
4. epistemic attacks
5. statistical disclosure attacks

However we also discuss a potential active confirmation attack for
reliable retransmission protocols (ARQ protocols) which are layered on
top of mix networks. In addition to that we use an anonymous,
asynchronous PAKE protocol for the initial key exchanges between
communication partners. The ACN used does augment the PAKE protocol's
threat model and this does contribute to the overall system design of
the catshadow messaging system.

The most important conclusion from the threat model analysis is that
decryption mix networks have a fundamental design disadvantage where
the sender always knows the final destination of their messages. This
design ``feature'' is in direct opposition to the goal of Sender
Receiver Unlinkability with Adversarial Sender. Anonymous messaging
systems designed for high risk use cases need to provide this and
other privacy notions in order to strongly hide user locations from
one another. The Catshadow messaging system overcomes this design
limitation of the Sphinx cryptographic packet format by having the
messages total path be composed of two routes, the first chosen by the
sender and the second route chosen by the receiver via a SURB reply.

Although there is a severe computational penalty, randomized
reencryption mix networks a different set of tradeoffs than our
messaging system for the privacy notions we desire. In particular,
reencryption mix networks by defaults always provide the privacy
notion Sender Receiver Unlinkability with Adversarial Sender. I'd like
to point out that although reencryption mixnets provide this privacy
notion without the added complexity which catshadow has. It is also
clear that using the Poisson mix strategy is essentially a commitment
to achieving high performance network throughput even if it means
degrading the anonymity guarantees to essentially having no protection
beyond what onion routing ACN's like Tor and I2p have to
offer. Therefore we can say that Catshadow has differing anonymity
properties depending on the amount of traffic flowing through the
network. Whereas reencryption mixnets are severely limited in their
ability to scale up the traffic capacity, however their threat model
is typically stronger and their anonymity properties do not vary like
Katzenpost does because of using the Poisson mix strategy.

\section{Acknowledgment}
I would like to thank Leif Ryge for his creative input in the design
of this mix network messaging system.

{\footnotesize \bibliographystyle{IEEEtran}
\bibliography{bibliography}
}


\appendices

\section{Bibtex Entry}
\label{FirstAppendix}

Note that the following bibtex entry is in the IEEEtran bibtex style
as described in a document called "How to Use the IEEEtran BIBTEX Style".
\vspace{8pt}
\begin{quote}
   @online\{CatshadowThreatModel,\\
   \hspace*{5pt} title = \{Catshadow Threat Model\},\\
   \hspace*{5pt} author = \{David Stainton\},\\
   \hspace*{5pt} url = \{https://sphinx.rs/papers/catshadow\_threat\_model/\},\\
   \hspace*{5pt} year = \{2020\}\\
   \}\\
\end{quote}

\end{document}
